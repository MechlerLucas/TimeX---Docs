%%%%%%%%%%%%%%%%%%%%%%%%%%%%%%%%%%%%%%%%%%%%%%%%%%%%%%%%%%%%%%%%%%%%%%%%%%%%%%%%%%%%%%%%%%%%%%%%%%%%%%%%%%%%%%%%%%%%%%%%%%%%%%%%
%%%%%%%%%%%%%% Template de Artigo Adaptado para Trabalho de Conclusão de Curso - SI Contagem - PUCMINAS                       %%
%% codificação UTF-8 - Abntex - Latex -  							                                                          %%
%% Autor da primeira versão:    Fábio Leandro Rodrigues Cordeiro                                                              %% 
%% Co-autores da primeira versão: Prof. João Paulo Domingos Silva, Harison da Silva e Anderson Carvalho		                  %%
%% Revisores normas NBR (Padrão PUC Minas) da primeira versão: Helenice Rego Cunha e Prof. Theldo Cruz                        %%
%% Versão: 1.1     18 de dezembro 2015                                                                                        %%
%%%%%%%%%%%%%%%%%%%%%%%%%%%%%%%%%%%%%%%%%%%%%%%%%%%%%%%%%%%%%%%%%%%%%%%%%%%%%%%%%%%%%%%%%%%%%%%%%%%%%%%%%%%%%%%%%%%%%%%%%%%%%%%%
\section{\esp Introdução} 

O Transtorno de Déficit de Atenção e Hiperatividade (TDAH), conforme discutido por \citeonline{carolo2009} citando \citeonline{barkley2008}, é um transtorno neurobiológico caracterizado por dificuldades nas funções executivas, como planejamento, organização e autorregulação, o que frequentemente resulta em prejuízos acadêmicos, profissionais e sociais. Nesse contexto, a área de aplicações móveis e ferramentas digitais de apoio vem se consolidando como uma alternativa promissora para auxiliar pessoas com esse perfil, oferecendo recursos que podem apoiar a organização pessoal de maneira prática e acessível.

Observa-se que muitas aplicações populares de produtividade apresentam excesso de funcionalidades, menus extensos e elementos visuais que podem contribuir para a perda de foco e abandono do uso. Estudos apontam que interfaces simples, limpas e com menos estímulos tendem a favorecer o uso contínuo por pessoas com dificuldades de atenção sustentada, conforme discutido por \citeonline{knouse2022}, \citeonline{almeida2024} e \citeonline{ferreira2023}. Assim, este trabalho parte da premissa de que soluções visuais minimalistas e processos reduzidos podem ser potencialmente mais adequados para usuários que enfrentam desafios de organização diária, em alinhamento com recomendações presentes na literatura.

A escolha deste tema justifica-se pela relevância social e acadêmica da proposta. Do ponto de vista social, observa-se que a maioria das aplicações populares de produtividade não é projetada para públicos neurodivergentes, e frequentemente apresenta excesso de menus, funções avançadas e telas densas. Para muitas pessoas com TDAH, esse excesso de elementos funciona como um fator de distração, contribuindo para a perda de foco e o abandono da ferramenta. Assim, a simplicidade torna-se não apenas desejável, mas uma estratégia essencial para reduzir a carga cognitiva durante o uso. Do ponto de vista acadêmico, este trabalho busca evidenciar como aplicações minimalistas podem ser potencialmente mais adequadas para usuários que apresentam dificuldades de foco e organização descritas na literatura sobre TDAH.

Em uma análise exploratória realizada em cinco aplicações populares de gerenciamento de tarefas: Todoist, Notion, Microsoft To Do, Trello e Google Keep, observou-se que embora sejam amplamente utilizados, essas aplicações tendem a adotar interfaces mais densas, com múltiplos menus, diversos ícones e fluxos longos de interação, o que pode dificultar o uso por pessoas com TDAH, conforme apontado na literatura. A ausência de mecanismos de categorização simplificada e de lembretes repetidos contrasta com recomendações de design voltadas a esse público, que destacam a importância de feedbacks visuais claros e estímulos controlados, para auxiliar na organização pessoal.

Portanto, este estudo pretende oferecer uma alternativa inspirada nas dificuldades relatadas na literatura e reforçadas por feedback exploratório obtido durante o desenvolvimento, como simplicidade visual, feedbacks imediatos, baixo número de etapas por ação, organização por cores e categorias claras e redução da sobrecarga cognitiva durante o uso da aplicação.

Nesse sentido, o objetivo geral deste trabalho é desenvolver uma aplicação web simples e responsivo de organização pessoal e gerenciamento de tempo, fundamentado em princípios de simplicidade, clareza visual e redução da carga cognitiva, elaborado com base nas dificuldades de organização e foco descritas na literatura sobre TDAH, buscando compreender como estratégias minimalistas podem auxiliar o usuário ao diminuir distrações e tornar a navegação mais direta.

São objetivos específicos deste trabalho:
(i) realizar uma revisão bibliográfica sobre TDAH e estratégias de organização pessoal; (ii) identificar, a partir da literatura, os requisitos funcionais e não funcionais para a aplicação; (iii) implementar funcionalidades essenciais evitando elementos que possam gerar sobrecarga cognitiva; (iv) desenvolver um protótipo navegável da aplicação; e (v) realizar uma avaliação preliminar utilizando de plataforma de avaliação de usabilidade.

%%%%%%%%%%%%%%%%%%%%%%%%%%%%%%%%%%%%%%%%%%%%%%%%%%%%%%%%%%%%%%%%%%%%%%%%%%%%%%%%%%%%%%%%%%%%%%%%%%%%%%%%%%%%%%%%%%%%%%%%%%%%%%%%

\section{\esp Referencial Teórico}

Nesta seção são apresentados os principais conceitos que fundamentam este trabalho. Serão discutidos os aspectos relacionados ao Transtorno de Déficit de Atenção e Hiperatividade (TDAH), as tecnologias assistivas digitais voltadas à organização pessoal e, por fim, os princípios de usabilidade aplicados ao desenvolvimento de aplicações claras e funcionais. O objetivo é fornecer uma base teórica que permita compreender como fatores psicológicos, tecnológicos e de design se conectam na criação de soluções digitais voltadas às dificuldades de organização associadas ao TDAH.

\subsection{\esp Neurodivergência e TDAH}

O conceito de neurodivergência refere-se à variação natural do funcionamento neurológico humano, englobando condições como Transtorno do Espectro Autista (TEA), dislexia e TDAH. Em vez de ser compreendida apenas como um déficit ou patologia, a neurodivergência enfatiza a diversidade cognitiva como parte da condição humana. Dentro desse espectro, de acordo com \citeonline{carolo2009}, o TDAH é um transtorno do neurodesenvolvimento caracterizado por sintomas persistentes de desatenção, impulsividade e, em alguns casos, hiperatividade. Esses sintomas se manifestam desde a infância e podem permanecer ao longo da vida adulta, gerando impactos significativos na vida acadêmica, social e profissional dos indivíduos.

Na literatura de \citeonline{barkley2008} é expresso que as dificuldades mais recorrentes no TDAH estão associadas ao controle inibitório, à memória de trabalho e à autorregulação emocional, habilidades fundamentais para planejar, organizar e executar atividades diárias. Por isso, é comum que pessoas com TDAH enfrentem problemas em cumprir prazos, gerenciar rotinas, manter o foco em tarefas longas e lidar com a procrastinação. Estudos como de \citeonline{knouse2022} reforçam que essas limitações não se explicam apenas pela falta de esforço ou disciplina, mas decorrem de diferenças estruturais e funcionais no cérebro, principalmente no córtex pré-frontal.

Segundo dados nacionais disponíveis em \citeonline{conitec2022}, a prevalência do Transtorno do Déficit de Atenção com Hiperatividade (TDAH) no Brasil é estimada em aproximadamente 7,6\% entre crianças e adolescentes de 6 a 17 anos, passando para cerca de 5,2\% na faixa etária de 18 a 44 anos e 6,1\% nos indivíduos com 44 anos ou mais. Em âmbito global, a Associação Brasileira do Déficit de Atenção estima que a prevalência situa-se entre 5\% e 8\% da população, o que reforça a relevância social e acadêmica de desenvolver ferramentas e estratégias de apoio voltadas a esse público.

No entanto, pesquisas recentes apontam que muitas aplicações populares de produtividade apresentam excesso de funcionalidades, menus complexos e estímulos visuais que podem prejudicar especialmente usuários que apresentam dificuldades de foco e organização, conforme discutido na literatura sobre TDAH. \citeonline{knouse2022} destacam que pessoas com TDAH tendem a abandonar ferramentas digitais quando a interface contém elementos supérfluos, pois a sobrecarga cognitiva gerada dificulta o foco e a continuidade do uso. De forma semelhante, \citeonline{almeida2024} argumentam que interfaces minimalistas, com poucos elementos por tela e fluxos curtos, favorecem a manutenção da atenção e reduzem distrações. \citeonline{ferreira2023} reforça que a simplicidade não é apenas uma preferência estética, mas um requisito essencial para reduzir a carga cognitiva e promover maior clareza durante o uso.

\subsection{\esp Tecnologias assistivas digitais}

As tecnologias assistivas compreendem um conjunto de recursos, serviços e dispositivos que buscam ampliar a autonomia e a qualidade de vida de pessoas com deficiências ou condições específicas. No Brasil, a Lei nº 13.146/2015 (Estatuto da Pessoa com Deficiência) define tecnologia assistiva como todo recurso, metodologia ou prática que possibilite compensar limitações funcionais, promovendo inclusão e participação social. No contexto do TDAH, essas tecnologias assumem uma dimensão particular, pois o foco recai sobre apoiar a organização pessoal, a gestão do tempo e a manutenção da atenção.

O avanço dos dispositivos móveis e a popularização dos smartphones ampliaram significativamente as possibilidades de intervenção digital. Aplicações de produtividade, agendas eletrônicas e sistemas de lembretes passaram a ser utilizadas como ferramentas compensatórias para pessoas com TDAH. Segundo \citeonline{knouse2022}, soluções digitais baseadas em terapia cognitivo-comportamental têm mostrado potencial para reduzir sintomas e aumentar a adesão a rotinas. \citeonline{ferreira2023} destaca que a simplificação das interfaces é fundamental para evitar sobrecarga cognitiva, enquanto \citeonline{ramos2024} exploram a gamificação como meio de tornar a experiência mais motivadora. Já \citeonline{almeida2024} reforçam a importância de elementos visuais limpos e acessíveis.

Além das aplicações voltadas especificamente ao TDAH, métodos tradicionais de organização também foram adaptados para o ambiente digital. O GTD \textit{(Getting Things Done)}, de \citeonline{allen2001}, e a Técnica Pomodoro, de \citeonline{cirillo2006}, são amplamente utilizados para melhorar a produtividade, embora apresentem limitações quando aplicados a pessoas com TDAH, devido à sua complexidade ou rigidez. Por esse motivo, adaptações como checklists simplificados, múltiplos alarmes e lembretes progressivos têm sido apontados como estratégias mais adequadas.

Assim, as tecnologias assistivas digitais configuram-se como um campo fértil para a inovação, especialmente quando direcionadas às dificuldades de organização associadas ao TDAH. Ao combinar princípios de produtividade com recursos tecnológicos adaptados, é possível criar soluções que ampliem a autonomia e a motivação dos usuários.

\subsection{\esp Usabilidade em aplicações para TDAH}

O conceito de usabilidade está associado à facilidade de uso de um sistema para que os usuários alcancem seus objetivos de forma eficaz, eficiente e satisfatória, segundo \citeonline{nielsen1994}. Ele propôs dez heurísticas fundamentais de usabilidade, que orientam o design centrado no usuário, a serem detalhadas nos resultados:

No contexto do TDAH, essas heurísticas ganham importância especial. Por exemplo, o design estético e minimalista e o reconhecimento em vez de memorização ajudam a reduzir a sobrecarga cognitiva; a prevenção de erros e o feedback imediato contribuem para manter a atenção do usuário; e a consistência visual favorece a previsibilidade das ações, reduzindo distrações e frustrações.

Assim, o desenvolvimento de aplicações voltadas a pessoas com TDAH deve alinhar-se a essas heurísticas, garantindo interfaces simples, coerentes e acessíveis.

Pesquisas recentes, como \citeonline{almeida2024}, indicam que o design de interfaces deve privilegiar a simplicidade, a consistência visual e a clareza de instruções. No contexto do TDAH, recomenda-se evitar excesso de estímulos, utilizar cores para organizar prioridades e criar lembretes múltiplos para uma mesma atividade. A remoção de elementos que pouco contribuem para a eficácia, como o botão de “soneca” em alarmes, também se mostra relevante, uma vez que pode induzir à procrastinação. Essas práticas visam reduzir a carga cognitiva do usuário, facilitando a interação com a aplicação.

Com base nessas recomendações, diversos princípios foram incorporados ao desenvolvimento do TimeX. A partir das heurísticas de \citeonline{nielsen1994} e das diretrizes de design para pessoas com TDAH discutidas por \citeonline{knouse2022}, \citeonline{ferreira2023} e \citeonline{almeida2024}, o protótipo adotou um conjunto de escolhas projetuais voltadas à redução da carga cognitiva. Entre elas, destacam-se o design minimalista, a organização das tarefas por categorias visuais simplificadas, a redução de menus e etapas de navegação, o fornecimento de feedbacks imediatos, mecanismos de prevenção de erros e a manutenção de consistência visual entre todas as telas. Tais decisões aproximam o protótipo das melhores práticas identificadas na literatura, reforçando sua adequação ao público-alvo.

Portanto, observa-se que a integração entre conceitos de neurodivergência, tecnologias digitais de apoio e usabilidade fornece um alicerce sólido para o desenvolvimento de soluções voltadas às dificuldades de organização associadas ao TDAH. O domínio desses fundamentos permite compreender as necessidades específicas discutidas na literatura e direcionar esforços de design e implementação para aplicações mais claras, funcionais e coerentes com essas recomendações.

Esses princípios de simplicidade, clareza e redução de elementos que possam gerar sobrecarga cognitiva orientam diretamente a proposta deste trabalho, que busca oferecer um protótipo funcional inspirado nas dificuldades descritas na literatura sobre TDAH.



%%%%%%%%%%%%%%%%%%%%%%%%%%%%%%%%%%%%%%%%%%%%%%%%%%%%%%%%%%%%%%%%%%%%%%%%%%%%%%%%%%%%%%%%%%%%%%%%%%%%%%%%%%%%%%%%%%%%%%%%%%%%%%%%

\section{\esp Trabalhos Relacionados}

O estudo de \citeonline{knouse2022} teve como objetivo avaliar a usabilidade e viabilidade do aplicativo móvel Inflow, baseado em terapia cognitivo-comportamental, para adultos com TDAH. A pesquisa foi conduzida em formato de estudo aberto de sete semanas, com 240 participantes recrutados online que responderam avaliações em diferentes pontos de tempo. Foram coletados relatos dos próprios usuários sobre sintomas, uso do aplicativo e satisfação com a experiência. Os resultados indicaram boa aceitação, uso frequente e percepção de redução nos sintomas e prejuízos relacionados ao TDAH, demonstrando viabilidade e recomendando novos estudos clínicos controlados. Tal proposta se aproxima deste trabalho pelo foco em soluções móveis voltadas a pessoas com TDAH, mas se diferencia por ter como base a terapia cognitivo-comportamental, enquanto o presente estudo foca na organização pessoal por meio de despertadores inteligentes e checklists.

A pesquisa de \citeonline{ferreira2023} buscou propor o desenvolvimento de uma aplicação para auxiliar pessoas com TDAH na gestão de tarefas. A metodologia envolveu revisão bibliográfica, análise de necessidades dos usuários e a prototipagem de um aplicativo móvel (APK) com fluxos de interação simplificados. O resultado foi a elaboração de um protótipo inicial, que mostrou potencial para facilitar a rotina de usuários com TDAH, embora o autor ressalte a necessidade de avaliações de usabilidade em etapas posteriores. Essa proposta se aproxima do presente trabalho por priorizar a simplicidade e a funcionalidade, mas difere ao desenvolver um aplicativo mobile nativo, enquanto o TimeX é uma aplicação web. Além disso, o trabalho de Ferreira não inclui recursos de alarmes inteligentes, que aqui são considerados centrais.

O estudo de \citeonline{almeida2024} teve como objetivo analisar como estratégias de design podem impulsionar a criação de aplicativos para TDAH. Para isso, foram utilizados métodos de revisão teórica, análise de recursos em aplicativos já existentes e aplicação de princípios de design centrado no usuário. Os resultados apontaram que interfaces limpas, estímulos visuais controlados e navegação simplificada são fatores determinantes para adesão e eficácia das ferramentas digitais. A semelhança com este trabalho está no foco na experiência do usuário e na simplicidade do design, mas difere-se por manter o design como objeto principal de análise, enquanto aqui a ênfase é na implementação prática de um aplicativo funcional.

Já o trabalho de \citeonline{ramos2024} teve como objetivo explorar a aplicação da gamificação como recurso auxiliar no tratamento de jovens e adultos com TDAH. A metodologia incluiu revisão bibliográfica e pesquisa de campo com 50 participantes, que responderam questionários sobre suas dificuldades e preferências. A partir desses dados, foi proposto o protótipo do aplicativo Quest Mind, que utiliza mecânicas de jogos para estimular motivação, organização e adesão ao tratamento. Os resultados apontaram boa aceitação da proposta e potencial para impactos positivos no tratamento. Essa iniciativa se assemelha ao presente trabalho pela busca de soluções digitais voltadas à organização de pessoas com TDAH, mas se distingue pela ênfase em gamificação como estratégia principal, enquanto este estudo propõe o uso de despertadores adaptados e checklists como instrumentos centrais.

%%%%%%%%%%%%%%%%%%%%%%%%%%%%%%%%%%%%%%%%%%%%%%%%%%%%%%%%%%%%%%%%%%%%%%%%%%%%%%%%%%%%%%%%%%%%%%%%%%%%%%%%%%%%%%%%%%%%%%%%%%%%%%%%

\section{Metodologia}

A metodologia adotada neste trabalho é de natureza aplicada e de desenvolvimento. O objetivo principal foi criar uma aplicação web funcional, denominada \textit{TimeX}, voltada à organização e ao gerenciamento de tempo de usuários que apresentam dificuldades de foco e planejamento descritas na literatura sobre TDAH.

\subsection{Classificação da pesquisa}

O estudo é classificado como aplicado, pois busca gerar uma solução prática fundamentada em bases teóricas. É também exploratório e descritivo, ao investigar as necessidades do público-alvo e descrever o processo de desenvolvimento da aplicação.

\subsection{Levantamento inicial com usuário}

Antes da implementação do protótipo, foi realizada uma consulta exploratória com um usuário diagnosticado com TDAH, com o objetivo de identificar dificuldades reais de organização, expectativas em relação à aplicação e funcionalidades consideradas úteis para seu cotidiano. As anotações obtidas nessa etapa contribuíram diretamente para a definição dos requisitos funcionais e não funcionais do sistema.

Durante a conversa, o usuário mencionou a necessidade de recursos que ajudassem a “fazer um pit stop para o cérebro”, indicando a importância de momentos estruturados para reorganizar prioridades. Também destacou o uso frequente de despertadores no celular, sugerindo que a aplicação deveria permitir alertas antecipados de uma tarefa, com notificações repetidas até o horário de execução, e recomendou a remoção da função soneca, por considerar que ela atrapalha sua atenção.

Outras sugestões incluíram: exibir notas da tarefa junto ao despertador; utilizar um sistema de cores para facilitar a identificação rápida das prioridades; criar checklists simples; e experimentar variações pequenas de horário em lembretes recorrentes (por exemplo, alertas entre 3h30 e 4h00 para compromissos das 4h00). O usuário também mencionou interesse em recursos como integração com assistentes virtuais, indicando caminhos potenciais para versões futuras do sistema.

Essas contribuições foram fundamentais para a definição do escopo inicial do TimeX, orientando a priorização de funcionalidades simples, visuais e focadas na redução da carga cognitiva.

\subsection{Etapas de desenvolvimento}

O processo de desenvolvimento foi dividido em etapas principais:

\begin{enumerate}[label=\alph*)]
    \item Revisão bibliográfica: Pesquisa em bases acadêmicas como Scielo, Google Scholar e ACM Digital Library, com foco em TDAH, tecnologias assistivas e usabilidade \cite{barkley2008,nielsen1994,norman2013,almeida2024,ferreira2023}.
    
    \item Definição de requisitos: Os requisitos funcionais e não funcionais foram definidos com base na literatura, na análise de aplicações similares e em contribuições obtidas por meio de uma consulta exploratória com um usuário diagnosticado com TDAH, priorizando simplicidade, clareza e redução de distrações.
    
    \item Design da interface: A interface foi projetada segundo os princípios de usabilidade de \citeonline{nielsen1994} e do design centrado no usuário de \citeonline{norman2013}, incorporando também recomendações específicas para o público com TDAH discutidas por \citeonline{almeida2024}, \citeonline{ferreira2023} e \citeonline{knouse2022}. Essas diretrizes enfatizam simplicidade visual, redução de estímulos desnecessários, navegação linear e organização clara das informações, elementos essenciais para minimizar a carga cognitiva desse público. Assim, adotou-se um design \textit{mobile-first} com layout limpo, contrastes controlados e categorização visual simplificada. Para garantir consistência e acessibilidade, foram utilizadas ferramentas como Tailwind CSS \cite{tailwindcss}, Shadcn/UI, Framer Motion e Lucide React, que permitiram criar uma interface leve, responsiva e alinhada às recomendações encontradas na literatura sobre usabilidade para usuários com TDAH.
    
    \item Implementação: O frontend foi desenvolvido em React.js \cite{react}, com gerenciamento de estado por React Query e navegação com React Router DOM. O backend foi construído localmente em Node.js com Express.js \cite{expressjs}, utilizando arquivos JSON para persistência de dados e estrutura de rotas REST.
    
    \item Testes e validação: Foram realizados testes exploratórios para verificar o funcionamento das rotas, da interface e das notificações do navegador. As funcionalidades foram avaliadas conforme os princípios de usabilidade e simplicidade identificados na literatura.
\end{enumerate}

Além disso, foi obtido um feedback exploratório de um usuário diagnosticado com TDAH, que testou o protótipo de forma informal, contribuindo com observações qualitativas consideradas posteriormente na discussão dos resultados.


% \subsection{Ambiente de desenvolvimento}

% O ambiente de desenvolvimento foi configurado localmente utilizando o Visual Studio Code e o Node.js. As dependências foram gerenciadas via NPM e o projeto estruturado em pastas separadas para frontend e backend. A Quadro~\ref{tab:tecnologias} apresenta as principais ferramentas utilizadas.


% \begin{quadro}[H]
% \centering
% \caption{Tecnologias utilizadas no desenvolvimento da aplicação web TimeX}
% \label{tab:tecnologias}
% \renewcommand{\arraystretch}{1.3}

% \footnotesize

% \begin{tabular}{p{4cm} p{10cm}}
% \hline
% Tecnologia/Ferramenta & Descrição resumida \\
% \hline
% React.js & Biblioteca JavaScript para construção de interfaces de usuário. \\
% \hline
% Tailwind CSS & Framework CSS utilitário para estilização responsiva. \\
% \hline
% Shadcn/UI & Conjunto de componentes acessíveis baseados em Tailwind CSS. \\
% \hline
% Framer Motion & Biblioteca de animação para React. \\
% \hline
% Lucide React & Pacote de ícones vetoriais leves. \\
% \hline
% React Router DOM & Gerenciador de rotas e navegação. \\
% \hline
% React Query & Gerenciamento de estado e cache de requisições assíncronas. \\
% \hline
% date-fns & Manipulação e formatação de datas. \\
% \hline
% Express.js & Framework para criação de APIs REST locais em Node.js. \\
% \hline
% Node.js & Ambiente de execução JavaScript no servidor. \\
% \hline
% Visual Studio Code & Ambiente de desenvolvimento integrado (IDE). \\
% \hline
% \end{tabular}
% \fonte{Elaborado pelos autores.}

% \end{quadro}



%%%%%%%%%%%%%%%%%%%%%%%%%%%%%%%%%%%%%%%%%%%%%%%%%%%%%%%%%%%%%%%%%%%%%%%%%%%%%%%%%%%%%%%%%%%%%%%%%%%%%%%%%%%%%%%%%%%%%%%%%%%%%%%%

\section{Resultados e Discussão}

Esta seção apresenta os principais resultados obtidos com o desenvolvimento da aplicação \textit{TimeX}, descrevendo sua implementação, suas características voltadas à redução da carga cognitiva e a avaliação heurística realizada a partir das diretrizes de \citeonline{nielsen1994}.

\subsection{Implementação do TimeX}

O \textit{TimeX} foi desenvolvido como uma aplicação web com arquitetura modular composta por uma interface em React.js e um backend local em Node.js e Express.js. A persistência de dados é realizada por meio de arquivos JSON, garantindo independência de servidores externos e facilitando a validação local do protótipo.

O processo de implementação seguiu os princípios do design centrado no usuário, \citeonline{norman2013}, priorizando simplicidade, clareza e redução de elementos visuais que possam gerar sobrecarga cognitiva. Tecnologias como Tailwind CSS, Shadcn/UI e Framer Motion foram empregadas para garantir consistência visual, responsividade e transições suaves. A categorização por cores, aplicada de forma controlada para evitar estímulos excessivos, foi utilizada para auxiliar o reconhecimento imediato. \citeonline{almeida2024}.

As Figuras~\ref{fig:tela-inicial}, \ref{fig:nova-tarefa} e \ref{fig:detalhes-tarefa}, respectivamente, apresentam as três principais telas da aplicação: Tela Inicial, Nova Tarefa e Detalhes da Tarefa, ilustrando a estrutura visual e os elementos fundamentais da interface.

\begin{figure}[H]
    \centering
    \begin{subfigure}{0.32\textwidth}
        \centering
        \includegraphics[width=0.95\linewidth]{figuras/Tela inicial.png}  
        \caption{Tela inicial}
        \label{fig:tela-inicial}
    \end{subfigure}
    \hfill
    \begin{subfigure}{0.32\textwidth}
        \centering
        \includegraphics[width=0.95\linewidth]{figuras/Nova Tarefa.png}  
        \caption{Nova tarefa}
        \label{fig:nova-tarefa}
    \end{subfigure}
    \hfill
    \begin{subfigure}{0.32\textwidth}
        \centering
        \includegraphics[width=0.95\linewidth]{figuras/Detalhes da Tarefa.png}  
        \caption{Detalhes da tarefa}
        \label{fig:detalhes-tarefa}
    \end{subfigure}
    \caption{Prints da aplicação TimeX}
    \label{fig:prints}
\end{figure}

Além da navegação simples, a aplicação conta com notificações via \textit{Web Notifications API} e alertas sonoros pela \textit{Web Audio API}, recursos que podem auxiliar usuários que enfrentam dificuldades de foco e organização a manter atenção e lembrar compromissos.

Por fim, todo o código-fonte do \textit{TimeX} foi disponibilizado em repositórios públicos no GitHub\footnote{\url{https://github.com/MechlerLucas/TimeX-App}}. 
A documentação completa também está disponível em um repositório próprio\footnote{\url{https://github.com/MechlerLucas/TimeX-Docs}}.


\subsection{Avaliação Heurística de Usabilidade}

Antes da análise heurística, apresenta-se, na Tabela~\ref{tab:lista-heuristicas}, a lista das dez heurísticas propostas por \citeonline{nielsen1994}, que serviram como referência para a avaliação das telas da aplicação.

\begin{quadro}[H]
\centering
\caption{Lista das 10 Heurísticas de Nielsen (1994)}
\footnotesize
\label{tab:lista-heuristicas}
\begin{tabular}{|p{1.2cm}|p{13cm}|}
\hline
\textbf{H1} & \textbf{Visibilidade do status do sistema:} Manter o usuário sempre informado sobre o que está acontecendo por meio de feedback apropriado. \\ \hline
\textbf{H2} & \textbf{Correspondência entre o sistema e o mundo real:} Utilizar linguagem, metáforas e elementos familiares ao usuário. \\ \hline
\textbf{H3} & \textbf{Controle e liberdade do usuário:} Permitir que o usuário desfaça ações e retorne sem prejuízo. \\ \hline
\textbf{H4} & \textbf{Consistência e padrões:} Manter uniformidade visual, textual e funcional em toda a interface. \\ \hline
\textbf{H5} & \textbf{Prevenção de erros:} Reduzir a probabilidade de erros e, quando possível, solicitar confirmações antes de ações críticas. \\ \hline
\textbf{H6} & \textbf{Reconhecimento em vez de memorização:} Tornar opções visíveis, minimizando a carga de memória do usuário. \\ \hline
\textbf{H7} & \textbf{Flexibilidade e eficiência de uso:} Atender tanto usuários iniciantes quanto experientes, oferecendo atalhos e uso ágil. \\ \hline
\textbf{H8} & \textbf{Design estético e minimalista:} Evitar excesso de informações, priorizando clareza e foco. \\ \hline
\textbf{H9} & \textbf{Ajuda no diagnóstico e correção de erros:} Mensagens de erro devem ser claras, indicando causas e soluções. \\ \hline
\textbf{H10} & \textbf{Ajuda e documentação:} Disponibilizar documentação clara, breve e acessível, quando necessária. \\ \hline
\end{tabular}
\end{quadro}

% O detalhamento abaixo sintetiza os resultados da análise heurística realizada a partir das diretrizes de \citeonline{nielsen1994}.

% \subsection*{Detalhamento das heurísticas por tela}
% \label{tab:avaliacao-heuristica}
% \subsubsection*{Tela Inicial}

% \begin{itemize}
%     \item \textbf{H1 – Visibilidade do status:} Presente na exibição da data atual, filtro selecionado e aba ativa na barra inferior.
%     \item \textbf{H2 – Correspondência com o mundo real:} Ícones universais como casa, calendário e “+”.
%     \item \textbf{H6 – Reconhecimento em vez de memorização:} Tarefas identificadas por cores e filtros visuais no topo.
%     \item \textbf{H7 – Flexibilidade e eficiência:} Filtros por cor permitem navegação rápida.
%     \item \textbf{H8 – Estética e design minimalista:} Layout limpo, com ênfase nos elementos essenciais.
% \end{itemize}

% \subsubsection*{Tela Nova Tarefa}

% \begin{itemize}
%     \item \textbf{H1 – Visibilidade do status:} Título “Nova Tarefa” e campos claramente identificados.
%     \item \textbf{H3 – Controle e liberdade:} Botão “Cancelar” permite interromper a ação.
%     \item \textbf{H4 – Consistência e padrões:} Elementos visuais seguem o padrão da tela inicial.
%     \item \textbf{H5 – Prevenção de erros:} Impede salvar sem título e destaca campos obrigatórios.
%     \item \textbf{H6 – Reconhecimento:} Seleção de categorias por cores dispensa memorização.
%     \item \textbf{H10 – Ajuda mínima:} Ícones e labels indicam claramente a função de cada campo.
% \end{itemize}

% \subsubsection*{Tela Detalhes da Tarefa}

% \begin{itemize}
%     \item \textbf{H2 – Correspondência com o mundo real:} Cores e informações que representam a categoria da tarefa.
%     \item \textbf{H3 – Controle e liberdade:} Ações de editar, excluir e retornar.
%     \item \textbf{H4 – Consistência:} Estrutura visual compatível com as demais telas.
%     \item \textbf{H6 – Reconhecimento:} Cor e ícone da categoria facilitam identificação.
%     \item \textbf{H8 – Estética minimalista:} Exibe apenas informações essenciais ao usuário.
% \end{itemize}

Além do resumo apresentado nesta seção, a análise heurística completa, incluindo a marcação visual das heurísticas diretamente sobre as telas da aplicação, encontra-se disponível como material complementar no repositório de documentação do projeto. O arquivo pode ser acessado em: \url{https://github.com/MechlerLucas/TimeX-Docs/blob/main/analise-heuristica.md}.


A análise heurística realizada sugere que o \textit{TimeX} apresenta boa aderência às heurísticas fundamentais, especialmente em relação à visibilidade do status (H1), ao uso de metáforas compreensíveis (H2), à consistência visual (H4), ao reconhecimento em vez de memorização (H6) e ao design minimalista (H8). Esses elementos indicam alinhamento inicial com recomendações voltadas à redução da sobrecarga cognitiva, conforme discutido na literatura sobre TDAH. Entretanto, por se tratar de uma avaliação qualitativa baseada somente na inspeção das telas, tais resultados devem ser interpretados como indicativos, e não conclusivos, uma vez que não foi realizada validação quantitativa ou testes controlados com usuários.

\subsubsection{Impressões iniciais de um usuário com TDAH}

Durante o desenvolvimento do protótipo, foram coletadas impressões iniciais de um usuário diagnosticado com TDAH, que já havia contribuído anteriormente na fase de levantamento de requisitos. Sua participação foi fundamental tanto para orientar funcionalidades essenciais quanto para avaliar a clareza e adequação do protótipo.

Entre os aspectos positivos, o usuário destacou o layout simples, a divisão clara das categorias e a organização visual das tarefas, elementos que facilitaram a compreensão das informações. No entanto, apontou que o alarme deveria ser mais intrusivo, uma vez que lembretes pouco chamativos tendem a passar despercebidos, especialmente para pessoas com TDAH. Esse ponto dialoga com suas sugestões iniciais sobre notificações repetidas e reforços mais evidentes.

Outro problema identificado foi a marcação automática de tarefas atrasadas como concluídas, o que dificultou a distinção entre atividades realmente realizadas e pendentes. O usuário sugeriu a inclusão de um estado intermediário, como “não feita” ou “atrasada”, bem como a possibilidade de reagendar automaticamente tarefas não concluídas para o dia seguinte. Ressaltou, contudo, que esse reagendamento deveria respeitar tarefas com periodicidade fixa, para evitar inconsistências.

Essas impressões reforçam pontos discutidos nas heurísticas de usabilidade e evidenciam oportunidades práticas de melhoria no protótipo, especialmente no que diz respeito à clareza dos feedbacks e ao comportamento das tarefas em diferentes estados.


\subsection{Discussão dos Resultados}

Os resultados sugerem que o \textit{TimeX} apresenta potencial para oferecer uma solução simples e direta, alinhada às necessidades descritas na literatura sobre usuários com dificuldades de foco e organização. A combinação entre categorização por cores, organização visual limpa e fluxos reduzidos facilita o foco, diminui distrações e torna o gerenciamento de tarefas mais previsível.

As funcionalidades implementadas dialogam com limitações frequentemente relatadas por pessoas com TDAH, como dificuldade de organização, esquecimento recorrente e necessidade de reforços visuais. A aderência ampla às heurísticas de \citeonline{nielsen1994} reforça a qualidade da interface e evidencia que aplicações minimalistas podem reduzir a carga cognitiva de forma mais eficiente do que ferramentas complexas de produtividade.

Assim, os resultados mostram que é possível conciliar simplicidade técnica, usabilidade e clareza visual em uma solução prática e aplicável.



%%%%%%%%%%%%%%%%%%%%%%%%%%%%%%%%%%%%%%%%%%%%%%%%%%%%%%%%%%%%%%%%%%%%%%%%%%%%%%%%%%%%%%%%%%%%%%%%%%%%%%%%%%%%%%%%%%%%%%%%%%%%%%%%

\section{Conclusão}

Este trabalho teve como objetivo desenvolver um protótipo simples de aplicação web de organização pessoal voltado ao apoio de usuários que apresentam dificuldades de foco e planejamento descritas na literatura sobre TDAH. A proposta buscou demonstrar que aplicações minimalistas podem ser mais adequadas para esse perfil de usuário quando comparados a ferramentas tradicionais de produtividade que apresentam excesso de funcionalidades.

O \textit{TimeX} foi implementado com foco na simplicidade visual, na clareza de uso e na redução da carga cognitiva durante as interações. As funcionalidades desenvolvidas, como alarmes configuráveis, categorização por cores e lembretes automáticos, foram projetadas com base em recomendações da literatura, buscando tornar o uso mais direto e reduzir elementos que possam gerar distrações. A interface adota uma abordagem minimalista e intuitiva. A avaliação heurística de usabilidade confirmou que o sistema apresenta boa conformidade com os princípios de design centrado no usuário, indicando que o protótipo é funcional e coerente com os objetivos definidos para esta etapa. Além disso, considerou-se o feedback exploratório de um usuário diagnosticado com TDAH, cujas impressões destacaram tanto aspectos positivos da experiência quanto sugestões de melhoria, reforçando pontos relevantes observados durante a análise.

Os resultados obtidos demonstraram compatibilidade com as expectativas iniciais, indicando que é possível unir simplicidade técnica e clareza visual em uma solução digital que apresenta potencial para apoiar usuários com desafios de organização. Apesar da ausência de testes empíricos com um grupo amplo de usuários, a análise heurística e o feedback exploratório permitiram validar a pertinência do protótipo e identificar oportunidades de aprimoramento, como a inclusão de confirmações de ação e mensagens de feedback mais detalhadas. Tais ajustes poderão tornar a experiência mais completa e robusta em versões futuras.

A principal contribuição desta pesquisa foi o desenvolvimento de uma ferramenta digital que aplica princípios de usabilidade e design minimalista inspirados na literatura sobre TDAH. O \textit{TimeX} representa um avanço inicial ao demonstrar como soluções simples podem ser estruturadas para apoiar usuários com dificuldades de foco e planejamento, contribuindo para a discussão sobre abordagens alternativas aos aplicativos de produtividade tradicionais. Além disso, o projeto evidencia que a aplicação de princípios de design centrado no usuário e heurísticas de usabilidade pode resultar em interfaces mais claras, consistentes e alinhadas às necessidades observadas na literatura.

Como trabalhos futuros, recomenda-se a realização de testes de usabilidade com usuários diagnosticados com TDAH, a fim de validar empiricamente as percepções levantadas na literatura e no feedback exploratório obtido. Sugere-se também o aprimoramento da aplicação com recursos adicionais, como sincronização em nuvem, notificações personalizadas e integração com assistentes virtuais. A continuidade deste projeto pode contribuir significativamente para o desenvolvimento de ferramentas digitais que apoiem a organização pessoal e promovam maior autonomia em contextos reais de uso.

